\documentclass{article}
\usepackage{comment}
\usepackage[english]{babel}
\usepackage[utf8]{inputenc}
\usepackage{fancyhdr}
\usepackage[round]{natbib}
\usepackage{graphicx}
\usepackage{url}
\usepackage{amsmath}
\usepackage{amssymb}
\DeclareMathOperator*{\argmax}{argmax}
\pagenumbering{arabic}

\pagestyle{fancy}
\fancyhf{}
\rhead{Mohammad Rahmani}
\lhead{Swarm intelligence}

\newcommand{\ignore}[1]{}
\begin{document}
	\bibliographystyle{plainnat}
	\title{Swarm intelligence}
	\author{Mohammad Rahmani}
	\date{}
	\maketitle
	Swarm intelligence describes the behavior of decentral-
	ized cooperative agents, whether natural or artificial, working
	toward a common global goal \cite{rizk-2018-decision-making-in-multiagent-systems-a-survey}[120]. Self-organized and distributed behavior of locally aware and locally interacting
	agents are pillars of swarm intelligence \cite{rizk-2018-decision-making-in-multiagent-systems-a-survey}[121].Systems mod-
	eled in this fashion generally consist of many autonomous but
	homogeneous agents implementing simple rules with agent
	interactions restricted to local neighborhoods.
	
	\paragraph{Insights}
		\subparagraph{Advantages}
	While such systems exhibit desirable properties like robustness, flexibility, scalability, low complexity, inherent parallelism, and fault tolerance \cite{brambilla-2013-swarm-robotics-a-review-from-the-swarm-engineering-perspective}, \cite{rizk-2018-decision-making-in-multiagent-systems-a-survey}[129], 
		\paragraph{Disadvantages} They have important
	limitations. Most swarm systems consist of identical agents,
	leading to their limitations according to \cite{rizk-2018-decision-making-in-multiagent-systems-a-survey}[129]. The agents must
	be homogeneous or can be divided into a small number of homogeneous clusters following simple rules to make decisions. However, there are many applications, such as search and rescue operations, that require heterogeneous, complex
	agents working toward a common goal.
	
	
	\paragraph{Similarity with RL}  In some ways, swarm intelligence is similar to RL; both
	are iterative algorithms that use a reinforcement signal to learn
	a solution \cite{rizk-2018-decision-making-in-multiagent-systems-a-survey}[121]
	
	
	\paragraph{Particle swarm optimization}
	\paragraph{Communication}
	\paragraph{Trajectory tracking}
	
	\section{Biology}
		\paragraph{Cells}
			\subparagraph{Migration}
			See footnote \footnote{\url{https://en.wikipedia.org/wiki/Collective_cell_migration}}
		\paragraph{Bees}
			\subparagraph{Path planning}
			Many algorithms have been inspired by bee colony behavior.
			Bee colony optimization \cite{rizk-2018-decision-making-in-multiagent-systems-a-survey}[122] is based on direct communication among agents performing a series of moves for a certain
			duration based on the strength or fitness of the solution, also
			known as “waggle dancing.” This recruits other agents to the
			most fit solution. Navigation is based on path integration where
			agents continuously update a vector indicating the position of
			the start location.
		\paragraph{Ants}
			\subparagraph{Path planning}
			Ant colony optimization (ACO), inspired
			by ant colony behavior, is a class of algorithms that rely on
			indirect communication \cite{rizk-2018-decision-making-in-multiagent-systems-a-survey}[123]. Navigation is based on deposit-
			ing pheromones along the trail. A more fit solution results
			in stronger pheromones on the trail that lead to recruiting
			more agents. Distributed
			implementations of ACO \cite{rizk-2018-decision-making-in-multiagent-systems-a-survey}[126], \cite{rizk-2018-decision-making-in-multiagent-systems-a-survey}[127]
		\paragraph{Birds}
			\subparagraph{}PSO is inspired by flocks of bird and schools of fish \cite{rizk-2018-decision-making-in-multiagent-systems-a-survey}[124]. Agents navigate the environment searching for better solutions using principles from birds’ movements. Distributed
				implementations of PSO \cite{rizk-2018-decision-making-in-multiagent-systems-a-survey}[128] have
				been developed to speedup convergence.
				
			\subparagraph{Pigeon Swarm optimization for path planning}
				A pigeon inspired optimization algorithm relied on the magnetic field, sun and landmarks to achieve path planning \cite{rizk-2018-decision-making-in-multiagent-systems-a-survey}[125] have
				been developed to speedup convergence.
			\subparagraph{Boids}
				See footnote \footnote{\url{https://en.wikipedia.org/wiki/Boids}}
	\section{Communication}
		Reference \footnote{\url{https://ieeexplore.ieee.org/document/4413635}}
	\bibliography{/home/donkarlo/Dropbox/projs/research/refs}
\end{document}