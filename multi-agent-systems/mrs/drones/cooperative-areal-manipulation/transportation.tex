\documentclass{article}
\usepackage{comment}
\usepackage[english]{babel}
\usepackage[utf8]{inputenc}
\usepackage{fancyhdr}
\usepackage[round]{natbib}
\usepackage{graphicx}
\usepackage{url}
\usepackage{amsmath}
\usepackage{amssymb}
\DeclareMathOperator*{\argmax}{argmax}
\DeclareMathOperator*{\argmin}{argmin}
\pagenumbering{arabic}
\usepackage{multicol}
\usepackage{siunitx}
\usepackage{soul}

\pagestyle{fancy}
\fancyhf{}
\rhead{Mohammad Rahmani}
\lhead{Cooperative aerial load transportation}

\newcommand{\ignore}[1]{}
\begin{document}
	\bibliographystyle{plainnat}
	\title{Cooperative aerial load transportation / Areal manipulation}
	\author{Mohammad Rahmani}
	\date{}
	\maketitle
	\section{Collective payload transportation in general}
		\cite{brambilla-2013-swarm-robotics-a-review-from-the-swarm-engineering-perspective} Fig 1 presents "collective transport" as behavior emerged from swarms.  
	\section{Motion planning}
		\paragraph{Swarms in general}
			\cite{medina-2018-robotic-swarm-motion-planning-for-load-carrying-and-manipulating}
		\paragraph{Taking the appropriate formation}
			\cite{vasarhelyi-2014-outdoor-flocking-and-formation-flight-with-autonomous-aerial-robots}
			\cite{soria-2020-swarmlab-a-matlab-drone-swarm-simulator}
		\paragraph {Collective decision making}
			\cite{han-2010-collective-decision-making-in-multi-agent-systems-by-implicit-leadership}
		\paragraph {Nature inspired}
			\cite{gelblum-2017-ant-groups-optimally-amplify-the-effect-of-transiently-informed-individuals}
		\paragraph{Trajectory}
			\subparagraph{Slalom}
			\cite{loianno-2017-cooperative-transportation-using-small-quadrotors-using-monocular-vision-and-inertial-sensing}\footnote{\url{https://www.youtube.com/watch?v=Ilrb_P-8od4} Time 1:13}
			\subparagraph{Altitude modification}
			\cite{loianno-2017-cooperative-transportation-using-small-quadrotors-using-monocular-vision-and-inertial-sensing}\footnote{\url{https://www.youtube.com/watch?v=Ilrb_P-8od4} Time 0.58}
		\paragraph{Obstacle avoidance}
			\cite{lee-2018-an-integrated-framework-for-cooperative-aerial-manipulators-in-unknown-environments}\footnote{\url{https://www.youtube.com/watch?v=kX1qzZJPmcE}} offers a total framework for stable flight with obstacles.
			\cite{spurny-2019-cooperative-transport-of-large-objects-by-a-pair-of-unmanned-aerial-systems-using-sampling-based-motion-planning} which carries rod-shape load between obstacles in real environment.
			\cite{loianno-2017-cooperative-transportation-using-small-quadrotors-using-monocular-vision-and-inertial-sensing}
			slalom path
		\paragraph{Algorithms}
			\subparagraph{Rapidly-exploring Random Tree \cite{lavalle-1998-rapidly-exploring-random-trees-a-new-tool-for-path-planning}}
			 \cite{spurny-2019-cooperative-transport-of-large-objects-by-a-pair-of-unmanned-aerial-systems-using-sampling-based-motion-planning} The proposed method uses a stochastic optimization with the aim to find a solution in which the states with distance between the UAVs close to the desired one are preferred which increases reliability and efficiency of the cooperative carrying task. 
	\section{Attachment type}
		\paragraph{Cable suspension}
		\subparagraph{From different points}
		\cite{michael-2009-cooperative-manipulation-and-transportation-with-aerial-robots} discusses lifting a triangle with three drones. 
		\cite{masone-2016-cooperative-transportation-of-a-payload-using-quadrotors-a-reconfigurable-cable-driven-parallel-robot} which discusses a cube suspended from different points by cables connected to multiple drones.
		\cite{mohiuddin-2020-energy-distribution-in-dual-uav-collaborative-transportation-through-load-sharing}
		\cite{jiang-2013-the-inverse-kinematics-of-cooperative-transport-with-multiple-aerial-robots} inverse kinematics of a triangle.
		\cite{sreenath-2013-dynamics-control-and-planning-for-cooperative-manipulation-of-payloads-suspended-by-cables-from-multiple-quadrotor-robots}
		\subparagraph{From single point}
		\cite{jackson-2020-scalable-cooperative-transport-of-cable-suspended-loads-with-uavs-using-distributed-trajectory-optimization}
		\cite{guerrero-2015-passivity-based-control-for-a-quadrotor-uav-transporting-a-cable-suspended-payload-with-minimum-swing}\footnote{url{https://www.youtube.com/watch?v=TQ9O4UVHmJM}}
		\cite{mohammadi-2020-control-of-multiple-quad-copters-with-a-cable-suspended-payload-subject-to-disturbances}
		\footnote{\url{https://www.youtube.com/watch?v=TQ9O4UVHmJM,https://www.youtube.com/watch?v=aQ9rBlenekU}}
		\cite{mohammadi-2018-decentralized-motion-control-in-a-cabled-based-multi-drone-load-transport-system}
		\cite{marina-2019-flexible-collaborative-transportation-by-a-team-of-rotorcraft}
		\cite{erskine-2019-wrench-analysis-of-cable-suspended-parallel-robots-actuated-by-quadrotor-unmanned-aerial-vehicles} \footnote{\url{https://www.youtube.com/watch?v=rOCvd4uHFV4}}
		\cite{thapa-2019-cooperative-aerial-load-transport-with-attitude-stabilization}
		\cite{xie-2020-towards-cooperative-transport-of-a-suspended-payload-via-two-aerial-robots-with-inertial-sensing} 
		
		\paragraph{Rigidly attached}
		\cite{mellinger-2010-cooperative-grasping-and-transport-using-multiple-quadrotors}
		\cite{loianno-2017-cooperative-transportation-using-small-quadrotors-using-monocular-vision-and-inertial-sensing}
		\cite{tagliabue-2017-robust-collaborative-object-transportation-using-multiple-mavs}\footnote{\url{https://www.youtube.com/watch?v=QvjyDV8sNPo}}
		\cite{mellinger-2010-cooperative-grasping-and-transport-using-multiple-quadrotors}\footnote{\url{https://www.youtube.com/watch?v=YBsJwapanWI}}
		\cite{nguyen-2015-aerial-tool-operation-system-using-quadrotors-as-rotating-thrust-generators}\footnote{\url{https://www.youtube.com/watch?v=lAYUR7LpDEU}} with round, spherical connectors. 
	
	\section{Payload shape}
		\paragraph{Shape un-aware}
			\cite{tagliabue-2017-robust-collaborative-object-transportation-using-multiple-mavs}
			\cite{mellinger-2010-cooperative-grasping-and-transport-using-multiple-quadrotors}
		\paragraph{Rod-shaped}
			\cite{tagliabue-2017-collaborative-transportation-using-mavs-via-passive-force-control}
			\cite{tagliabue-2017-robust-collaborative-object-transportation-using-multiple-mavs}'s first experiment is performed by such tubes.
			\cite{gassner-2017-dynamic-collaboration-without-communication-vision-based-cable-suspended-load-transport-with-two-quadrotors}
			\cite{bandala-2018-payload-lift-and-transport-using-decentralized-unmanned-aerial-vehicle-quadcopter-teams}
			\cite{wu-2020-cooperative-transportation-of-drones-without-inter-agent-communication}
			\cite{mohiuddin-2020-energy-distribution-in-dual-uav-collaborative-transportation-through-load-sharing} which also covers un-equal load \footnote{\url{https://www.youtube.com/watch?v=o2K4BWfRGHk&t=219s}}.
			\cite{spurny-2019-cooperative-transport-of-large-objects-by-a-pair-of-unmanned-aerial-systems-using-sampling-based-motion-planning}
			\cite{lee-2018-an-integrated-framework-for-cooperative-aerial-manipulators-in-unknown-environments}\footnote{\url{https://www.youtube.com/watch?v=kX1qzZJPmcE}}
		\paragraph{Regular polygon}
			in both \cite{tagliabue-2017-collaborative-transportation-using-mavs-via-passive-force-control}
			and \cite{tagliabue-2017-robust-collaborative-object-transportation-using-multiple-mavs}, the system is shape-unaware. 
			\cite{jiang-2013-the-inverse-kinematics-of-cooperative-transport-with-multiple-aerial-robots} inverse kinematics of a triangle.
			\cite{sreenath-2013-dynamics-control-and-planning-for-cooperative-manipulation-of-payloads-suspended-by-cables-from-multiple-quadrotor-robots}
		\paragraph{Cylinder}
			\cite{mohammadi-2018-decentralized-motion-control-in-a-cabled-based-multi-drone-load-transport-system}
		\paragraph{Cube}
			\cite{masone-2016-cooperative-transportation-of-a-payload-using-quadrotors-a-reconfigurable-cable-driven-parallel-robot}
	\section{Formation control}
		\paragraph{Centralized}
			\cite{nguyen-2015-aerial-tool-operation-system-using-quadrotors-as-rotating-thrust-generators}
			\subparagraph{Using motion capture systems}
			\paragraph{Decentralized}
			
			\paragraph{Leader follower}
			\cite{tagliabue-2017-robust-collaborative-object-transportation-using-multiple-mavs}
			\cite{tagliabue-2017-collaborative-transportation-using-mavs-via-passive-force-control}
			\cite{gassner-2017-dynamic-collaboration-without-communication-vision-based-cable-suspended-load-transport-with-two-quadrotors}
			\cite{loianno-2017-cooperative-transportation-using-small-quadrotors-using-monocular-vision-and-inertial-sensing}
			\cite{wu-2020-cooperative-transportation-of-drones-without-inter-agent-communication}
			\cite{chen-2019-cooperative-transportation-of-cable-suspended-slender-payload-using-two-quadrotors}
			\cite{xie-2020-towards-cooperative-transport-of-a-suspended-payload-via-two-aerial-robots-with-inertial-sensing} follower only uses its IMU information derived from cable's force which connects it to the leader.
		\paragraph{Leaderless}
			\cite{mohammadi-2018-decentralized-motion-control-in-a-cabled-based-multi-drone-load-transport-system,mohammadi-2020-control-of-multiple-quad-copters-with-a-cable-suspended-payload-subject-to-disturbances}
			\cite{rezaee-2017-almost-sure-attitude-consensus-in-multispacecraft-systems-with-stochastic-communication-links}
	\section{Communication}
		\paragraph{Explicit} Each robot, uses other robots measurements.
			\cite{loianno-2017-cooperative-transportation-using-small-quadrotors-using-monocular-vision-and-inertial-sensing}\footnote{\url{https://www.youtube.com/watch?v=Ilrb_P-8od4}}
			\cite{masone-2016-cooperative-transportation-of-a-payload-using-quadrotors-a-reconfigurable-cable-driven-parallel-robot}
			\cite{spurny-2019-cooperative-transport-of-large-objects-by-a-pair-of-unmanned-aerial-systems-using-sampling-based-motion-planning}\footnote{\url{https://www.youtube.com/watch?v=Pdg3j791I9c&feature=youtu.be}}\footnote{\url{https://www.youtube.com/watch?v=FQH769AnYbQ&feature=youtu.be}} the planning task is realized onboard of one of the robots and the obtained trajectories are shared (using Wi-Fi in the presented HW setup) with its neighbor to ensure reliable coordination of the system.
			
		\paragraph{Implicit} Implicitly by sensing the internal forces of the robots acting on the transported object
			\cite{tagliabue-2017-robust-collaborative-object-transportation-using-multiple-mavs}
			\cite{tagliabue-2017-collaborative-transportation-using-mavs-via-passive-force-control}
			\cite{gassner-2017-dynamic-collaboration-without-communication-vision-based-cable-suspended-load-transport-with-two-quadrotors}\footnote{\url{https://www.youtube.com/watch?v=8pFBufXOumw}}
			\cite{wu-2020-cooperative-transportation-of-drones-without-inter-agent-communication} \footnote{\url{https://www.youtube.com/watch?v=ib9IxCNAVPs}}
			\cite{thapa-2019-cooperative-aerial-load-transport-with-attitude-stabilization}
		\paragraph{No communication}
			\cite{mohammadi-2018-decentralized-motion-control-in-a-cabled-based-multi-drone-load-transport-system,mohammadi-2020-control-of-multiple-quad-copters-with-a-cable-suspended-payload-subject-to-disturbances} uses a motion capture camera to coordinate. 
	\section{Energy aware}
		\cite{mohiuddin-2020-energy-distribution-in-dual-uav-collaborative-transportation-through-load-sharing}
	
	\section{Surveys}
		\cite{chung-2018-a-survey-on-aerial-swarm-robotics}
		\cite{coppola-2020-a-survey-on-swarming-with-micro-air-vehicles-fundamental-challenges-and-constraints}
		\cite{villa-2019-a-survey-on-load-transportation-using-multirotor-uavs} discusses particularly load transportation with multi-rotor UAVs 
		\cite{ruggiero-2018-introduction-to-the-special-issue-on-aerial-manipulation} discusses exclusively Areal manipulation
		\cite{tuci-2018-cooperative-object-transport-in-multi-robot-systems-a-review-of-the-state-of-the-art} also discusses exclusively areal transportation. 
		\cite{mohiuddin-2020-a-survey-of-single-and-multi-uav-aerial-manipulation} also discusses exclusively areal manipulation for both single and multiple agents. 
	\section{Environment}
		\paragraph{Static}
		\paragraph{Dynamic}
		\cite{spurny-2019-cooperative-transport-of-large-objects-by-a-pair-of-unmanned-aerial-systems-using-sampling-based-motion-planning}
		\cite{mora-2017-multi-robot-formation-control-and-object-transport-in-dynamic-environments-via-constrained-optimization}
	\section{Grasping}
		\paragraph{Magnetic}
			\cite{loianno-2017-cooperative-transportation-using-small-quadrotors-using-monocular-vision-and-inertial-sensing}
			\cite{mellinger-2010-cooperative-grasping-and-transport-using-multiple-quadrotors}
		\paragraph{friction-based}
		\paragraph{Penetration based}
		\paragraph{Spherical joints}
			\cite{tagliabue-2017-robust-collaborative-object-transportation-using-multiple-mavs}
	
	\section{Projects}
		\cite{ollero-2018-the-aeroarms-project-aerial-robots-with-advanced-manipulation-capabilities-for-inspection-and-maintenance}
	\section{Others}
	\section{Per paper}
		\paragraph{\cite{spurny-2019-cooperative-transport-of-large-objects-by-a-pair-of-unmanned-aerial-systems-using-sampling-based-motion-planning}}
			\begin{itemize}
				\item \textbf{Motion planing} The environment is not dynamic. That is obstacles are already known and the motion is already planned.
					\begin{itemize}
						\item \textbf{Probabilistic Roadmaps (PRM)}: \cite{lavalle-1998-rapidly-exploring-random-trees-a-new-tool-for-path-planning}
						\item \textbf{Rapidly Exploring Random Trees (RRT)}: \cite{kavraki-1996-probabilistic-roadmaps-for-path-planning-in-high-dimensional-configuration-spaces}
					\end{itemize}
				\item \textbf{Communication} Wifi
				\item GPSS for localization
			\end{itemize}
		\paragraph{\cite{loianno-2017-cooperative-transportation-using-small-quadrotors-using-monocular-vision-and-inertial-sensing}}
			\begin{itemize}
				\item \textbf{Sensors}: 
					\begin{itemize}
						\item IMU
						\item Camera
					\end{itemize}
				\item \textbf{Covers}
					\begin{itemize}
						\item Estimation
						\item Control
						\item Trajectory planning
					\end{itemize}
				\item \textbf{Key contributions}
					\begin{itemize}
						\item a new approach to coordinated control, which allows independent control of each vehicle while guaranteeing the system’s stability
						\item a new cooperative localization scheme that allows each vehicle to benefit from measurements acquired by other vehicles
					\end{itemize}
			\end{itemize}
		\paragraph{\cite{tagliabue-2017-robust-collaborative-object-transportation-using-multiple-mavs}}
		Highlights
		\begin{itemize}
			\item Multiple MAVs mechanically coupled.
			\item Uses spherical joints
			\item doesn’t require force/torque sensors
			\item decentralized
			\item doesn’t rely on communication between agents and can be easily extended to multiple MAVs
			\item  spherical joint, which guarantees attitude decoupling among agents and payload
			\item kinetically constraints the translational dynamic of each agent to be the same as the motion of the payload
			\item The task of the master is to lead and steer the group of vehicles connected to the payload towards a desired destination, while keeping a constant transportation altitude.
			\item The slaves are not aware of the destination goal, but share the same altitude of the master, assisting their leader in the transportation task.
			\item  Cooperation and coordination of the slave agents with respect to the movements of the master is achieved via re-
			shaping the apparent physical properties of each slave, in order that maximum compliance to the actions of the master is guaranteed.
			\item modify the inertial properties of each slave MAV to make it behave as a \textbf{passive point-mass},
			accelerated by any interaction force applied on it and subject to viscous friction directly proportional to its own velocity. This new behavior can be achieved by means of \textbf{admittance control}.
			\item The key idea behind admittance control is that a \textbf{position-controlled mechanical system} can achieve arbitrary
			interaction dynamics by sensing the force coming from the environment and by accordingly generating and following
			a reference trajectory. In order to do so, every slave agent is equipped with a force estimator, which senses the force
			applied by the master to the payload. This force information is then used by the admittance controller to regulate the
			position of the slave agent, so that it guarantees maximum possible compliance to the estimated force.
			\item In a collaborative transportation maneuver, the master MAV agents behaves as if it was carrying the payload
			alone, pulling it towards the desired goal while keeping a \textbf{constant altitude}. For this reason it only executes its on-board \textbf{reference tracking feedback loop}. The reference trajectory which leads it to the destination goal can be provided by a user or by an on-board or off-board planning system.
			\item Information is instead shared via the payload itself, which is used as a medium to transfer the interaction force
			applied by the master to the slaves. This information flow is mono-directional, from the master to the slaves, as the slaves
			fully act according to the sensed force applied by the master on the transported structure, while the master executes its
			standard reference tracking algorithms.
			\item agents (a) do not have to agree on a \textbf{global inertial reference frame} 
			\item do not need to share the destination goal. This is made possible by the fact that each slave on-line generates
			a reference trajectory in the same frame as the external forces are estimated. The destination goal has to be known only by
			the master and, as a consequence, has to be expressed in the master’s reference frame only.
			
			\item the agents can grasp the payload in such a way that its weight force can be equally distributed among the
			agents.
			
		\end{itemize}
		\paragraph{Reaching the transportation altitude / taking off and landing}
			\begin{itemize}
				\item A centralized Finite State Machine (FSM) sends altitude increment commands $\Delta h$ to both the agents and make
				sure they both fulfill the command.
			\end{itemize}
		\paragraph{Agents low level architecture}
			\subparagraph{State estimator}
			\subparagraph{Position and attitude controller}
		\paragraph{Slave-specific control architecture}
			\subparagraph{Force estimator}
				\begin{itemize}
					\item Force estimates can be obtained from the sole inertial information provided by the on-board IMU and the state estimator
					\item combined with the measurement of the rotational speed of the rotors
				\end{itemize}
			\subparagraph{Admittance controller}
				 generates an on-line reference trajectory (position, velocity, attitude and angular velocity) for the position and attitude controller, given the estimate of the external force.
	\section{Academic events}
		\paragraph{Conferences}
		\paragraph{Journals}
			\begin{itemize}
				\item Journal of advanced transportation \footnote{\url{https://www.hindawi.com/journals/jat/}} 
			\end{itemize}
	\bibliography{/home/donkarlo/Dropbox/projs/research/refs.bib}
\end{document}